\textbf{Escoge 2 ideas o conceptos que te parezcan interesantes o controversiales y elabora sobre ellos. (300 palabras mínimo)}\vspace{.2cm}

\textcolor{bibi}{¿Porque las PyMEs son clave para el la computación cuántica en México?}
\begin{quote}
	Lo primero que se puede pensar cuando se hace esta pregunta es, ¿porque serían las empresas chicas y medianas mas importantes que las empresas grandes y los gobiernos? aunque es cierto que estas ultimas cuentan con muchos mas recursos, la verdad es que en la práctica suelen ser bastante lentas en adoptar el cambio y evitan proyectos que se pueden ver como riesgosos. \vspace{.3cm}

	Otro punto a considerar en este aspecto, es que aunque es cierto que el hardware cuántico presenta un costo considerable, existe una gran oportunidad de negocio cuando se trata de software o incluso de capacitación humana para temas cuánticos. Para estas areas se pueden hacer uso de herramientas que hacen del computo cuántico mucho mas abierto, como lo es el computo cuántico en la nube que ya ofrecen todos los hiperescaladores. \vspace{.3cm}

	Como adicional, es claro que si las PyMEs comienzan a experimentar con este tipo de tecnologías y consiguen casos de uso útiles, entonces se estimularía la oferta laboral y se podría crear una cadena de valor interesante induciendo un ciclo vicioso que beneficiaria tanto económicamente como tecnológicamente. \vspace{.3cm}

	Finalmente, es importante destacar que aunque existe un mercado potencial alto, a día de hoy es bastante complicado llegar a utilizar el computo cuántico en algo universal y son mas bien los nichos que pueden ser explotados por empresas chicas flexibles y que no necesitan contratos millonarios.
\end{quote}


\textcolor{bibi}{Necesitamos entender mas sobre el comportamiento cuántico antes de que sea económicamente viable.}
\begin{quote}
	Una idea muy prevalente en el texto es el hecho de que se necesitan mejores estándares para el computo cuántico y aunque es una idea muy buena y hasta natural, algo que ha sido claro para mi como estudiante de la materia, es lo intrínsicamente complejos que son los comportamientos cuánticos y el computo asociado. \vspace{.3cm}

	Desde algo tan sencillo como explicar que es un qubit o una operacián sobre el mismo, tratar de interiorizar el que esta pasando me deja pensando sobre la viabilidad de que alguien ademas de los expertos puedan dar buen uso de esta tecnología emergente. \vspace{.3cm}

	Aunque aun es pronto para saber que deparara el futuro del computo cuántico, creo que es esencial que se haga una preparación al menos en factor humano sobre este tema para tener disponible si es que este ámbito llega a tener una popularidad como la que tenemos actualmente con la IAG.
\end{quote}
