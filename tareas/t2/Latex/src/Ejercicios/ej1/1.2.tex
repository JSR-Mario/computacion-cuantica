\textbf{Adicionalmente, explica de manera detallada la estrategia que debe seguir Bob para
transformar su qubit al estado $\ket{\psi}$.}\vspace{.2cm}

\textcolor{bibi}{Propiedades}
\begin{quote}
    Para esta parte voy a mostrar las propiedades que utilice de los operadores de Pauli:
    \begin{align*}
        \hat{\sigma}_x =
        \begin{pmatrix}
            0 & 1\\
            1 & 0
        \end{pmatrix}
        , \ \ \ \ket{\psi} = 
        \begin{pmatrix}
            \alpha\\
            \beta
        \end{pmatrix}
    \end{align*}

    \begin{align*}
        \hat{\sigma}_x\ket{\psi} = 
        \begin{pmatrix}
            0*\alpha +1*\beta\\
            1*\alpha +0*\beta
        \end{pmatrix}
        = 
        \begin{pmatrix}
            \beta\\
            \alpha
        \end{pmatrix}
    \end{align*}

    Vemos que evidentemente la $\hat{\sigma}_x$ no hace mas que la inversion. Y ahora para la
    $\hat{\sigma}_z$ vamos a probar que invierte la fase:

    \begin{align*}
        \hat{\sigma}_z =
        \begin{pmatrix}
            1 & 0\\
            0 & -1
        \end{pmatrix}
        , \ \ \ \ket{\psi} = 
        \begin{pmatrix}
            \alpha\\
            \beta
        \end{pmatrix}
    \end{align*}

     \begin{align*}
        \hat{\sigma}_z\ket{\psi} = 
        \begin{pmatrix}
            1*\alpha +0*\beta\\
            0*\alpha -1*\beta
        \end{pmatrix}
        = 
        \begin{pmatrix}
            \alpha\\
            -\beta
        \end{pmatrix}
    \end{align*}

    Por tanto los pasos que describimos para recuperar el qubit original dependiendo del caso son
    validos. 
\end{quote}
