\textbf{Use las relaciones anteriores para reconstruir $\rho$ únicamente a partir de
$\braket{X}, \braket{Y},\braket{Z}$.} \vspace{.3cm}

Usamos las siguientes identidad trigonométricas:

$$
\cos^2{\frac{\theta}{2}} = \frac{1+\cos{\theta}}{2}, \ \ \ \sin^2{\frac{\theta}{2}} =
\frac{1-\cos{\theta}}{2}, \ \ \ 2cos \frac{\theta}{2} sin \frac{\theta}{2} = sin \theta
$$

Sustituyendo 
$$
\rho =
\begin{pmatrix}
\frac{1+\cos{\theta}}{2} &  \frac{1}{2} sin \theta \ e ^{-i \phi} \\
\frac{1}{2} sin \theta \ e ^{i \phi} & \frac{1-\cos{\theta}}{2}
\end{pmatrix}
$$ \vspace{.3cm}

Como $\braket{X}=sin \theta cos \phi$, $\braket{Y}=sin \theta sin \phi$, $\braket{Z}=cos \theta$,
obtenemos:

\begin{align*}
\rho &= \frac{1}{2}
\begin{pmatrix}
1+\cos{\theta} & sin \theta \ e ^{-i \phi} \\
sin \theta \ e ^{i \phi} & 1-\cos{\theta}
\end{pmatrix} \\ 
     &= \frac{1}{2}
\begin{pmatrix}
1+\braket{Z} & sin \theta \ e ^{-i \phi} \\
sin \theta \ e ^{i \phi} & 1-\braket{Z}
\end{pmatrix} \\
&= \frac{1}{2}
\begin{pmatrix}
1+\braket{Z} & sin \theta \ (cos \phi -i sin \phi) \\
sin \theta \ (cos \phi +i sin \phi) & 1-\braket{Z}
\end{pmatrix} \\
&= \frac{1}{2}
\begin{pmatrix}
1+\braket{Z} & sin (\theta) cos (\phi) - i \ sin (\theta) sin (\phi) \\
sin (\theta) cos (\phi) + i \ sin (\theta) sin (\phi) & 1-\braket{Z}
\end{pmatrix} \\
&= \frac{1}{2}
\begin{pmatrix}
1+\braket{Z} & \braket{X} - i \braket{Y} \\
\braket{X} + i \ \braket{Y} & 1-\braket{Z}
\end{pmatrix} \\
\end{align*}
