\textbf{Encuentre los autovectores normalizados de Y.} \vspace{.3cm}

\begin{itemize}
    \item \textbf{Autovector para $\lambda_1 = 1$} \vspace{.3cm}

        Formamos Y-$\lambda_1 I$:
            $$
            Y-\lambda_1 I =
            \begin{pmatrix}
            -1 & -i\\
            i & -1
            \end{pmatrix}
            $$

        Resolver $(Y-I)v=0$. Si $v=(x,y)^T$:
        $$\begin{cases}
            -x-yi = 0 \\
            xi -y = 0
        \end{cases}$$

        Tomando la segunda ecuación tenemos que $y=ix$ sustituyendo en la primera $-x-i(ix)=0$
        tenemos que $0=0$ por lo que las ecuaciones son dependientes. \vspace{.3cm}

        De la segunda ecuación podemos escribir cualquier vector solución como:
        $$v=
        \begin{pmatrix}
            x \\
            ix
        \end{pmatrix}=x
        \begin{pmatrix}
            1 \\
            i
        \end{pmatrix}
        $$

        Si v es autovector, cv tambien lo es para cualquier c diferente de 0, tomamos a $x=1$.
        \vspace{.3cm}

        Normalizamos el vector:

        $$
        ||v|| = \sqrt{|1|^2+|i|^2} = \sqrt{1+1} = \sqrt{2}
        $$

        Por tanto el vector normalizado es:

        $$\frac{1}{\sqrt{2}}
        \begin{pmatrix}
            1 \\
            i
        \end{pmatrix}
        $$

        Verificamos que sea correcto: \vspace{.3cm}

        $$
        Yv = \begin{pmatrix}
            0 & -i\\
            i & 0
            \end{pmatrix}
        \frac{1}{\sqrt{2}}
        \begin{pmatrix}
            1 \\
            i
        \end{pmatrix}=\frac{1}{\sqrt{2}}
        \begin{pmatrix}
            1 \\
            i
        \end{pmatrix}=1*v=\lambda_1 *v
        $$
        \vspace{.3cm}

    \item \textbf{Autovector para $\lambda_2=-1$}\vspace{.3cm}

        Formamos Y-$\lambda_2 I$:
            $$
            Y-\lambda_2 I =
            \begin{pmatrix}
            1 & -i\\
            i & 1
            \end{pmatrix}
            $$

        Resolver $(Y+I)v=0$. Si $v=(x,y)^T$:
        $$\begin{cases}
            x-yi = 0 \\
            xi+y = 0
        \end{cases}$$

        Tomando la primera ecuación tenemos $x=yi$ sustituyendo en la segunda $(yi)i+y=0$ tenemos
        entonces que $-y+y=0$ y $0=0$ aplicamos lo mismo.\vspace{.3cm}

        $$v=
        \begin{pmatrix}
            yi \\
            y
        \end{pmatrix}=y
        \begin{pmatrix}
            i \\
            1
        \end{pmatrix}
        $$

        Si v es autovector, cv tambien lo es para cualquier c diferente de 0, tomamos a $y=1$.
        \vspace{.3cm}

        Normalizamos el vector:

        $$
        ||v|| = \sqrt{|i|^2+|1|^2} = \sqrt{1+1} = \sqrt{2}
        $$

        Por tanto el vector normalizado es:

        $$\frac{1}{\sqrt{2}}
        \begin{pmatrix}
            i \\
            1
        \end{pmatrix}
        $$

        Verificamos que sea correcto: \vspace{.3cm}

        $$
        Yv = \begin{pmatrix}
            0 & -i\\
            i & 0
            \end{pmatrix}
        \frac{1}{\sqrt{2}}
        \begin{pmatrix}
            i \\
            1
        \end{pmatrix}=\frac{1}{\sqrt{2}}
        \begin{pmatrix}
            -i \\
            -1
        \end{pmatrix}=- \frac{1}{\sqrt{2}}
        \begin{pmatrix}
            i \\
            1
        \end{pmatrix}
        =-1*v=\lambda_2 *v
        $$
        \vspace{.3cm}
\end{itemize}
