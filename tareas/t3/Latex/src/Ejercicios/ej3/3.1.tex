\textbf{Calcule explícitamente la matriz unitaria}

$$U(t)=e^{-iXt}$$

$$
X =
\begin{pmatrix}
0 & 1\\
1 & 0
\end{pmatrix}, \ \ \ X^2 = I
$$

Usando la expansion de Taylor de la exponencial matricial:

$$
e^{-iXt} = \sum_{n=0}^\infty \frac{(-it)^n X^n}{n!}
$$

Ahora separamos los términos pares e impares, aprovechamos la propiedad de la matriz de Pauli: 
\begin{itemize}
    \item $(n=2k): X^{2k} = I$
    \item $(n=2k+1): X^{2k+1} = X$
\end{itemize}

\begin{align*}
    U(t) = I \sum_{k=0}^{\infty} \frac{(-1)^kt^{2k}}{(2k)!}-iX \sum_{k=0}^{\infty}
    \frac{(-1)^kt^{2k+1}}{(2k+1)!}
\end{align*}

Estas son las series de Taylor para seno y coseno:
\begin{align*}
    U(t) = I cos (t) - i X sin(t)
\end{align*}

Usando las definiciones matriciales:
\begin{align*}
    U(t) &= \begin{pmatrix}
1 & 0\\
0 & 1
\end{pmatrix} cos (t) - i \begin{pmatrix}
0 & 1\\
1 & 0
\end{pmatrix} sin(t) \\ 
&= \begin{pmatrix}
cos (t) & 0\\
0 & cos (t)
\end{pmatrix}+ \begin{pmatrix}
0 & -i \ sin(t)\\
-i \ sin(t) & 0
\end{pmatrix} \\
&= \begin{pmatrix}
cos (t) & -i \ sin(t)\\
-i \ sin(t) & cos (t)
\end{pmatrix}
\end{align*}

\vspace{.3cm}
