\textbf{Se nos promete que exatcamente una de cuatro cajas contiene un regalo. Las cajas están
etiquetadas por $x \in \{0,1\}^2$. Definamos la función oráculo.}\vspace{.2cm}
\begin{align*}
    f : \{0,1\}^2 \Rightarrow \{0,1\}, && f(x) = \begin{cases}
        1 & \text{ si la caja x contiene el regalo} \\
        0 & \text{ en otro caso} \\
    \end{cases}
\end{align*}

\textbf{Se nos dice que es posible averiguar en qué caja está el regalo con un solo llamado de la función
$f$. Calcule el circuito de la figura y justifique por qué éste permite conocer la caja con el
regalo con las mediciones mostradas.}

\begin{center}
    \includegraphics[width=.6\textwidth]{src/Img/fig1.png}
\end{center}

\textcolor{bibi}{}
\begin{quote}
\end{quote}
